Velik broj problema može se modelirati grafom u kojem vrhovi predstavljaju stanja, a bridovi prijelaze između tih stanja, takav graf se naziva prostor stanja \engl{state space}.
Rješavanje problema se onda svodi na pretraživanje prostora stanja, odnosno pronalazak najkraćeg puta između početnog stanja i stanja koje predstavlja rješenje problema.

% Problem pronalaska najkraćeg puta između dva vrha u grafu jedan je od najstarijih i najbolje istraženih problema u teoriji grafova. Rješenja tog problema imaju široku primjenu jer se velik broj situacija u pravom svijetu može modelirati kao grafovi, gdje vrhovi predstavljaju stanja, a bridovi prijelaze između tih stanja. Provedbom ovih algoritama minimizira se broj prijelaza, odnosno trošak za prijelaz od početnog do završnog stanja.

Zbog svoje široke primjenjivosti, razni algoritmi su osmišljeni za efikasno rješavanje tog problema. U ovom seminaru razmatrat će se prednosti i mane različitih algoritama za pronalazak najkraćeg puta između dva vrha u grafu, počevši od jednostavnijih naivnih algoritama,  do složenijih za implementaciju, ali efikasnijih informiranih algoritama s naglaskom na algoritam A* i njegovu primjenu na pronalazak najkraćeg puta u cjelobrojnoj rešetci. 

U svrhu jednostavne vizualizacije rada algoritama, koristit će se cjelobrojna rešetka u kojoj sivi kvadratići predstavljaju neprolazna polja, dok bijeli predstavljaju polja kroz koja se može proći. Moguće je kretati se u 4 osnovna smjera. "A" predstavlja početno stanje (vrh), a "B" završno. 

\begin{figure}[h]
	\centering
	
\begin{tikzpicture}
	\begin{scope}
		\begin{scope}[xshift=0.25cm, yshift=-0.25cm]
	\draw[step=0.5cm,black,very thin] (-2,-2) grid (2,2);
	
	\newcommand\fillSquare[2]{\filldraw[fill=lightgray] (#1 / 2,#2 / 2) rectangle (#1 / 2 + 0.5,#2 / 2 + 0.5)}
	
	
	\foreach \x in {-3,...,2}
	{
		\fillSquare{\x}{-3};
		\fillSquare{\x}{2};
		\fillSquare{2}{\x};
	}
\end{scope}

\matrix[matrix, matrix of nodes, nodes={anchor=center,inner sep=0pt,text width=.5cm,align=center,minimum height=.5cm}, nodes in empty cells]{
& 0 & 1 & 2 & 3 & 4 & 5 & 6 & 7 \\
 0 &   &   &   &   &   &   &   & B \\
	1 &   &   &   &   &   &   &   &   \\
	2 &   &   &   &   &   &   &   &   \\
	3 &   &   &   &   &   &   &   &   \\
	4 & A &   &   &   &   &   &   &   \\
	5 &   &   &   &   &   &   &   &   \\
	6 &   &   &   &   &   &   &   &   \\
	7 &   &   &   &   &   &   &   &   \\};
	\end{scope}
	
	\begin{scope}[xshift=7cm]
	    \matrix[matrix, matrix of nodes, nodes={anchor=center,inner sep=0pt,text width=.5cm,align=center,minimum height=.5cm}, nodes in empty cells]{
   	& 0 & 1 & 2 & 3 & 4 & 5 & 6 & 7 \\
   	0 &   &   &   &   &   &   &   &   \\
   	1 &   &   &   &   &   &   &   &   \\
   	2 &   &   &   &   &   &   &   &   \\
   	3 &   &   &   &   &   &   &   &   \\
   	4 &   &   &   &   &   &   &   &   \\
   	5 &   &   &   &   &   &   &   &   \\
   	6 &   &   &   &   &   &   &   &   \\
   	7 &   &   &   &   &   &   &   &   \\};
  	
\newcommand\drawNode[4]{\node[circle,draw,inner sep=0pt,minimum size=0.4cm] at (#1 / 2 - 1.5, #2 / 2 - 2)   (#4) {#3};}
\newcommand\drawWallNode[3]{\node[circle,draw,fill=lightgray,inner sep=0pt,minimum size=0.4cm] at (#1 / 2 - 1.5, #2 / 2 - 2)   (#3) {};}

\drawNode{0}{0}{ }{0}
\drawNode{1}{0}{ }{1}
\drawNode{2}{0}{ }{2}
\drawNode{3}{0}{ }{3}
\drawNode{4}{0}{ }{4}
\drawNode{5}{0}{ }{5}
\drawNode{6}{0}{ }{6}
\drawNode{7}{0}{ }{7}
\drawNode{0}{1}{ }{8}
\drawWallNode{1}{1}{9}
\drawWallNode{2}{1}{10}
\drawWallNode{3}{1}{11}
\drawWallNode{4}{1}{12}
\drawWallNode{5}{1}{13}
\drawWallNode{6}{1}{14}
\drawNode{7}{1}{ }{15}
\drawNode{0}{2}{ }{16}
\drawNode{1}{2}{ }{17}
\drawNode{2}{2}{ }{18}
\drawNode{3}{2}{ }{19}
\drawNode{4}{2}{ }{20}
\drawNode{5}{2}{ }{21}
\drawWallNode{6}{2}{22}
\drawNode{7}{2}{ }{23}
\drawNode{0}{3}{A}{24}
\drawNode{1}{3}{ }{25}
\drawNode{2}{3}{ }{26}
\drawNode{3}{3}{ }{27}
\drawNode{4}{3}{ }{28}
\drawNode{5}{3}{ }{29}
\drawWallNode{6}{3}{30}
\drawNode{7}{3}{ }{31}
\drawNode{0}{4}{ }{32}
\drawNode{1}{4}{ }{33}
\drawNode{2}{4}{ }{34}
\drawNode{3}{4}{ }{35}
\drawNode{4}{4}{ }{36}
\drawNode{5}{4}{ }{37}
\drawWallNode{6}{4}{38}
\drawNode{7}{4}{ }{39}
\drawNode{0}{5}{ }{40}
\drawNode{1}{5}{ }{41}
\drawNode{2}{5}{ }{42}
\drawNode{3}{5}{ }{43}
\drawNode{4}{5}{ }{44}
\drawNode{5}{5}{ }{45}
\drawWallNode{6}{5}{46}
\drawNode{7}{5}{ }{47}
\drawNode{0}{6}{ }{48}
\drawWallNode{1}{6}{49}
\drawWallNode{2}{6}{50}
\drawWallNode{3}{6}{51}
\drawWallNode{4}{6}{52}
\drawWallNode{5}{6}{53}
\drawWallNode{6}{6}{54}
\drawNode{7}{6}{ }{55}
\drawNode{0}{7}{ }{56}
\drawNode{1}{7}{ }{57}
\drawNode{2}{7}{ }{58}
\drawNode{3}{7}{ }{59}
\drawNode{4}{7}{ }{60}
\drawNode{5}{7}{ }{61}
\drawNode{6}{7}{ }{62}
\drawNode{7}{7}{B}{63}
\draw[] (0) -- (8);
\draw[] (0) -- (1);
\draw[] (1) -- (2);
\draw[] (1) -- (0);
\draw[] (2) -- (3);
\draw[] (2) -- (1);
\draw[] (3) -- (4);
\draw[] (3) -- (2);
\draw[] (4) -- (5);
\draw[] (4) -- (3);
\draw[] (5) -- (6);
\draw[] (5) -- (4);
\draw[] (6) -- (7);
\draw[] (6) -- (5);
\draw[] (7) -- (15);
\draw[] (7) -- (6);
\draw[] (8) -- (16);
\draw[] (8) -- (0);
\draw[] (15) -- (23);
\draw[] (15) -- (7);
\draw[] (16) -- (24);
\draw[] (16) -- (8);
\draw[] (16) -- (17);
\draw[] (17) -- (25);
\draw[] (17) -- (18);
\draw[] (17) -- (16);
\draw[] (18) -- (26);
\draw[] (18) -- (19);
\draw[] (18) -- (17);
\draw[] (19) -- (27);
\draw[] (19) -- (20);
\draw[] (19) -- (18);
\draw[] (20) -- (28);
\draw[] (20) -- (21);
\draw[] (20) -- (19);
\draw[] (21) -- (29);
\draw[] (21) -- (20);
\draw[] (23) -- (31);
\draw[] (23) -- (15);
\draw[] (24) -- (32);
\draw[] (24) -- (16);
\draw[] (24) -- (25);
\draw[] (25) -- (33);
\draw[] (25) -- (17);
\draw[] (25) -- (26);
\draw[] (25) -- (24);
\draw[] (26) -- (34);
\draw[] (26) -- (18);
\draw[] (26) -- (27);
\draw[] (26) -- (25);
\draw[] (27) -- (35);
\draw[] (27) -- (19);
\draw[] (27) -- (28);
\draw[] (27) -- (26);
\draw[] (28) -- (36);
\draw[] (28) -- (20);
\draw[] (28) -- (29);
\draw[] (28) -- (27);
\draw[] (29) -- (37);
\draw[] (29) -- (21);
\draw[] (29) -- (28);
\draw[] (31) -- (39);
\draw[] (31) -- (23);
\draw[] (32) -- (40);
\draw[] (32) -- (24);
\draw[] (32) -- (33);
\draw[] (33) -- (41);
\draw[] (33) -- (25);
\draw[] (33) -- (34);
\draw[] (33) -- (32);
\draw[] (34) -- (42);
\draw[] (34) -- (26);
\draw[] (34) -- (35);
\draw[] (34) -- (33);
\draw[] (35) -- (43);
\draw[] (35) -- (27);
\draw[] (35) -- (36);
\draw[] (35) -- (34);
\draw[] (36) -- (44);
\draw[] (36) -- (28);
\draw[] (36) -- (37);
\draw[] (36) -- (35);
\draw[] (37) -- (45);
\draw[] (37) -- (29);
\draw[] (37) -- (36);
\draw[] (39) -- (47);
\draw[] (39) -- (31);
\draw[] (40) -- (48);
\draw[] (40) -- (32);
\draw[] (40) -- (41);
\draw[] (41) -- (33);
\draw[] (41) -- (42);
\draw[] (41) -- (40);
\draw[] (42) -- (34);
\draw[] (42) -- (43);
\draw[] (42) -- (41);
\draw[] (43) -- (35);
\draw[] (43) -- (44);
\draw[] (43) -- (42);
\draw[] (44) -- (36);
\draw[] (44) -- (45);
\draw[] (44) -- (43);
\draw[] (45) -- (37);
\draw[] (45) -- (44);
\draw[] (47) -- (55);
\draw[] (47) -- (39);
\draw[] (48) -- (56);
\draw[] (48) -- (40);
\draw[] (55) -- (63);
\draw[] (55) -- (47);
\draw[] (56) -- (48);
\draw[] (56) -- (57);
\draw[] (57) -- (58);
\draw[] (57) -- (56);
\draw[] (58) -- (59);
\draw[] (58) -- (57);
\draw[] (59) -- (60);
\draw[] (59) -- (58);
\draw[] (60) -- (61);
\draw[] (60) -- (59);
\draw[] (61) -- (62);
\draw[] (61) -- (60);
\draw[] (62) -- (63);
\draw[] (62) -- (61);
\draw[] (63) -- (55);
\draw[] (63) -- (62);
		
			
			
	\end{scope}
\end{tikzpicture}
	\caption{Primjer jednostavne cjelobrojne rešetke i njezinog prostora stanja.} 
\end{figure}



