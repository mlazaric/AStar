Brojni problemi se mogu prikazati skupom stanja i prijelazima između njih, odnosno prostorom stanja. 
Rješavanje problema se svodi na istraživanje prostora stanja, te pronalazak najkraćeg puta iz nekog početnog stanja, do stanja koje predstavlja rješenje.

Postoje različiti algoritmi za istraživanje prostora stanja. 
Naivni algoritmi istražuju prostor stanja naivno, bez dodatnih informacija o problemu, što dovodi do velike vremenske i memorijske složenosti.

Informirani algoritmi koriste dodatne informacije kako bi smanjili prostor stanja koji trebaju istražiti.
Te dodatne informacije se ugrađuju u heurističku funkciju koja predstavlja aproksimaciju najmanje cijene prijelaza od stanja \( n \) do stanja koje predstavlja rješenje.
Prednost informiranih algoritama je manja vremenska i memorijska složenost zato što ne moraju istražiti cijeli prostor stanja.

Algoritam A* je informirani algoritam za pretraživanje prostora stanja koji kombinira heurističku funkciju i trenutnu cijenu kako bi smanjio prostor stanja kojeg treba istražiti, ali svejedno našao optimalno rješenje.
Postoje varijante algoritma A* koje nastoje smanjiti memorijsku složenost kao što su IDA* i SMA*.

U ovom seminaru opisane su obje kategorije algoritama, i naivni i informirani, prikazane su njihove razlike, kako u implementaciji, tako i u memorijskoj i vremenskoj složenosti. Osim same analize algoritama, vizualizacijama je pobliže objašnjen način rada, kao i slučajevi u kojima opisani algoritmi ne daju dobre rezultate.