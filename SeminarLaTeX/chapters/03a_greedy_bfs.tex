Pretraživanje "najbolji prvi" \engl{greedy best-first-search} je jednostavan pohlepan informiran algoritam koji nije optimalan, a potpun je jedino ako koristimo listu posjećenih stanja. Vremenska i prostorna složenost mu je \( O(b^m) \). \cite{russelNorvig2003:aima} \cite{umjetna}

Algoritam je ekvivalentan pretraživanju s jednolikom cijenom, osim što za evaluacijsku funkciju koristi heurističku funkciju \( f(n) = h(n) \).
To znači da će prvo evaluirati stanja koja su bliža rješenju, no taj lokalni optimum neće nužno dovesti do optimalnog rješenja, kao što je prikazano na slici \ref{greedy}.

\begin{figure}[h]
	\centering
	\begin{tikzpicture}
		\begin{scope}
			\begin{scope}[xshift=0.25cm, yshift=-0.25cm]
	\fill[fill = green!10] (-2, -0.5) rectangle (2, 0);
	
	\draw[step=0.5cm,black,very thin] (-2,-2) grid (2,2);
\end{scope}

\matrix[matrix, matrix of nodes, nodes={anchor=center,inner sep=0pt,text width=.5cm,align=center,minimum height=.5cm}, nodes in empty cells]{
	  & 0 & 1 & 2 & 3 & 4 & 5 & 6 & 7 \\
	0 &   &   &   &   &   &   &   &   \\
	1 &   &   &   &   &   &   &   &   \\
	2 &   &   &   &   &   &   &   &   \\
	3 &   &   &   &   &   &   &   &   \\
	4 & A & 1 & 2 & 3 & 4 & 5 & 6 & B \\
	5 &   &   &   &   &   &   &   &   \\
	6 &   &   &   &   &   &   &   &   \\
	7 &   &   &   &   &   &   &   &   \\
};
		\end{scope}
		
		\begin{scope}[xshift = 7.5cm]
			\begin{scope}[xshift=-1.75cm, yshift=1.75cm, xscale=0.5, yscale=0.5]
	\fill[fill = lightgray] (1, -1) -- ++(6, 0) -- ++(0, -6) -- ++(-6, 0) -- ++(0, 1) -- ++(5, 0) -- ++(0, 4) -- ++(-5, 0) -- cycle;
	
	\fill[fill = green!10] (0, -5)
	  -- ++(0, 1)
	  -- ++(5, 0)
	  -- ++(0, 1)
	  -- ++(-5, 0)
	  -- ++(0, 3)
	  -- ++(8, 0)
	  -- ++(0, -1)
	  -- ++(-7, 0)
	  -- ++(0, -1)
	  -- ++(5, 0)
	  -- ++(0, -3)
	  -- cycle;
	  
	\fill[fill = red!10] (1, -4)
	  -- ++(0, 1)
	  -- ++(4, 0)
	  -- ++(0, -1)
	  -- cycle;
	  
	\fill[fill = red!10] (3, -6)
	  -- ++(0, 1)
	  -- ++(3, 0)
  	  -- ++(0, -1)
	  -- cycle;
\end{scope}

\begin{scope}[xshift=0.25cm, yshift=-0.25cm]	
	\draw[step=0.5cm,black,very thin] (-2,-2) grid (2,2);
\end{scope}

\matrix[matrix, matrix of nodes, nodes={anchor=center,inner sep=0pt,text width=.5cm,align=center,minimum height=.5cm}, nodes in empty cells]{
	  & 0 & 1 & 2 & 3 & 4 & 5 & 6 & 7 \\
	0 & 14 & 15 & 16 & 17 & 18 & 19 & 20 & B \\
	1 & 13 &   &   &   &   &   &   &   \\
	2 & 12 & 11 & 10 & 9 & 8 & 7 &   &   \\
	3 &   & 8 & 5 & 6 & 9 & 6 &   &   \\
	4 & A & 1 & 2 & 3 & 4 & 5 &   &   \\
	5 &   &   &   & 4 & 7 & 6 &   &   \\
	6 &   &   &   &   &   &   &   &   \\
	7 &   &   &   &   &   &   &   &   \\
};
		\end{scope}
	\end{tikzpicture}
	\caption{Rezultat pohlepnog pretraživanja.} 
	\label{greedy}
\end{figure}