Algoritam A* je ekvivalentan pretraživanju s jednolikom cijenom, osim što za evaluacijsku funkciju koristi sumu cijene puta od početnog stanja do stanja \( n \) i heurističke funkcije \( f(n) = g(n) + h(n) \), odnosno aproksimaciju najboljeg puta kroz stanje \( n \). 
A* je uvijek potpun, a optimalan je pod uvjetom da je heuristika \( h(n) \) konzistentna.
Vremenska i memorijska složenost mu je \( O \left (\textrm{min}(b^{d + 1}, b|S|) \right ) \) gdje \( |S| \) predstavlja ukupan broj stanja. \cite{russelNorvig2003:aima} \cite{umjetna}


\begin{figure}[h]
	\centering
	\begin{tikzpicture}
		\begin{scope}
			\begin{scope}[xshift=0.25cm, yshift=-0.25cm]
	\fill[fill = green!10] (-2, -0.5) rectangle (2, 0);
	
	\draw[step=0.5cm,black,very thin] (-2,-2) grid (2,2);
\end{scope}

\matrix[matrix, matrix of nodes, nodes={anchor=center,inner sep=0pt,text width=.5cm,align=center,minimum height=.5cm}, nodes in empty cells]{
	& 0 & 1 & 2 & 3 & 4 & 5 & 6 & 7 \\
	0 &   &   &   &   &   &   &   &   \\
	1 &   &   &   &   &   &   &   &   \\
	2 &   &   &   &   &   &   &   &   \\
	3 &   &   &   &   &   &   &   &   \\
	4 & A & 1 & 2 & 3 & 4 & 5 & 6 & B \\
	5 &   &   &   &   &   &   &   &   \\
	6 &   &   &   &   &   &   &   &   \\
	7 &   &   &   &   &   &   &   &   \\
};
		\end{scope}
		
		\begin{scope}[xshift = 7.5cm]
			\begin{scope}[xshift=-1.75cm, yshift=1.75cm, xscale=0.5, yscale=0.5]
	\fill[fill = lightgray] (1, -1) -- ++(6, 0) -- ++(0, -6) -- ++(-6, 0) -- ++(0, 1) -- ++(5, 0) -- ++(0, 4) -- ++(-5, 0) -- cycle;
	
	\fill[fill = green!10] (1, -5)
	  -- ++(-1, 0)
	  -- ++(0, 5)
	  -- ++(8, 0)
	  -- ++(0, -1)
	  -- ++(-7, 0)
	  -- cycle;
	  
	\fill[fill = red!10] (1, -5)
	  -- ++(0, 3)
	  -- ++(5, 0)
	  -- ++(0, -1) -- ++(-1, 0)
	  -- ++(0, -1) -- ++(-1, 0)
	  -- ++(0, -1) -- ++(-1, 0)
	  -- cycle;
\end{scope}

\begin{scope}[xshift=0.25cm, yshift=-0.25cm]	
	\draw[step=0.5cm,black,very thin] (-2,-2) grid (2,2);
\end{scope}

\matrix[matrix, matrix of nodes, nodes={anchor=center,inner sep=0pt,text width=.5cm,align=center,minimum height=.5cm}, nodes in empty cells]{
	  & 0 & 1 & 2 & 3 & 4 & 5 & 6 & 7 \\
	0 & 4 & 5 & 6 & 7 & 8 & 9 & 10 & B \\
	1 & 3 &   &   &   &   &   &   &   \\
	2 & 2 & 3 & 4 & 5 & 6 & 7 &   &   \\
	3 & 1 & 2 & 3 & 4 & 5 &   &   &   \\
	4 & A & 1 & 2 & 3 &   &   &   &   \\
	5 &   &   &   &   &   &   &   &   \\
	6 &   &   &   &   &   &   &   &   \\
	7 &   &   &   &   &   &   &   &   \\
};
		\end{scope}
	\end{tikzpicture}
	\caption{Rezultat izvođenja algoritma A*.} 
	\label{astar}
\end{figure}

Takva evaluacijska funkcija kombinira prednosti pretraživanja s jednolikom cijenom i pretraživanja "najbolji prvi".
Zbog heurističke funkcije algoritam preferira stanja koja se približavaju cilju, a zbog funkcije cijene algoritam je optimalan (uz konzistentnu heuristiku) jer preferira stanja koja imaju najjeftiniji put od početnog stanja do cilja kroz to stanje.

Postoje brojne varijante algoritma A*.
Algoritam A* s iterativnim povećanjem dubine \engl{iterative-deepening A*, IDA*} je ekvivalentan pretraživanju u dubinu s iterativnim povećavanjem dubine, no umjesto dubine, IDA* koristi evaluacijsku funkciju \( f(n) \). 
Prednost algoritma IDA* je manja memorijska složenost, a posebno je praktičan za grafove s uniformnim težinama bridova. \cite{russelNorvig2003:aima}

Pojednostavljeni algoritam A* s ograničenom memorijom \engl{simplified memory-bound A*, SMA*} se ponaša kao običan A* sve dok ne napuni ograničenu memoriju. 
Kada ju napuni, algoritam izbacuje list s najvećom vrijednosti evaluacijske funkcije.
Vrijednost izbačenog vrha se zapisuje u njegovog roditelja, kako bi znao je li potrebno ponovno obrađivati taj vrh. 
SMA* je optimalan ako optimalno rješenje stane u ograničenu memoriju. \cite{russelNorvig2003:aima}