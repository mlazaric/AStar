Opisani problem definiran je s pet komponenti \cite{russelNorvig2003:aima}:

\begin{enumerate}
	\item Početno stanje \engl{initial state} 
	\item Moguće akcije \engl{actions} -- opis mogućih akcija u nekom stanju.
	\item Model prijelaza \engl{transition model} opis što svaka akcija znači. 
	\item Provjera riješenosti \engl{goal test} -- provjera je li neko stanje rješenje.
	\item Funkcija troška prijelaza između stanja \engl{step cost}
\end{enumerate}

Za problem cjelobrojne rešetke, početno stanje je stanje sa slovom "A".
Moguće akcije su: gore, dolje, lijevo i desno.
Model prijelaza je intuitivan: gore povećava y koordinatu za jedan, dolje ju smanjuje, a lijevo i desno povećavaju, odnosno smanjuju x koordinatu.
Provjera riješenosti provjerava sadrži li stanje slovo "B".
Trošak neposrednog prijelaza između stanja je uvijek jednak 1.
Definicija tog problema implementirana je razredom \texttt{RectangularGrid}.