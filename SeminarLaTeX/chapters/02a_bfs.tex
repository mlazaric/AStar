Pretraživanje u širinu \engl{breadth-first search} je jednostavan algoritam za pronalazak najkraćeg puta u grafu s uniformnim težinama bridova.
Algoritam je potpun i optimalan, a njegova vremenska i memorijska složenost je \( O(b^d) \).

Za pohranu vrhova koje treba obraditi koristi red \engl{queue}, dok za pohranu već obrađenih vrhova koristi skup \engl{set}.
U svakom koraku širi se u svim mogućim smjerovima, što dovodi do velike memorijske i vremenske složenosti.
Implementiran je funkcijom \texttt{bfs} u razredu \texttt{SearchAlgorithms}.

Na slici \ref{inefficient_bfs} je prikazano izvođenje algoritma.
Crvenom bojom su označena polja koje je algoritam obradio, ali nisu dovela do najkraćeg puta, dok je zelenom označen najkraći put.
Brojevi u poljima označavaju udaljenost od početnog polja.
Pretraga u širinu je obradila \( 44 \) polja, dok bi optimalno rješenje obradilo samo \( 8 \). 

\begin{figure}[h]
	\centering
	
\begin{tikzpicture}
	\begin{scope}
		\begin{scope}[xshift=0.25cm, yshift=-0.25cm]
			\fill[fill = green!10] (-2, -0.5) rectangle (2, 0);
			
			\path[fill = red!10] (-2, -0.5) -- (1.5, -0.5) -- (1.5, -1) -- (1, -1) -- (1, -1.5) -- (0.5, -1.5) -- (0.5, -2) -- (-2, -2) -- cycle;
			
			\path[fill = red!10] (-2, 0) -- ++(0, 2) -- ++(1.5, 0) -- ++(0, -0.5) -- ++(0.5, 0) -- ++(0, -0.5) -- ++(0.5, 0) -- ++(0, -0.5) -- ++(0.5, 0) -- ++(0, -0.5) -- cycle;
		
			\draw[step=0.5cm,black,very thin] (-2,-2) grid (2,2);
		\end{scope}
		
		\matrix[matrix, matrix of nodes, nodes={anchor=center,inner sep=0pt,text width=.5cm,align=center,minimum height=.5cm}, nodes in empty cells]{
			  & 0 & 1 & 2 & 3 & 4 & 5 & 6 & 7 \\
			0 & 4 & 5 & 6 &   &   &   &   &   \\
			1 & 3 & 4 & 5 & 6 &   &   &   &   \\
			2 & 2 & 3 & 4 & 5 & 6 &   &   &   \\
			3 & 1 & 2 & 3 & 4 & 5 & 6 &   &   \\
			4 & A & 1 & 2 & 3 & 4 & 5 & 6 & B \\
			5 & 1 & 2 & 3 & 4 & 5 & 6 & 7 &   \\
			6 & 2 & 3 & 4 & 5 & 6 & 7 &   &   \\
			7 & 3 & 4 & 5 & 6 & 7 &   &   &   \\
		};
	\end{scope}
	
	\begin{scope}[xshift = 7.5cm]
		\begin{scope}[xshift=0.25cm, yshift=-0.25cm]
			\fill[fill = green!10] (-2, -0.5) rectangle (2, 0);
			
			\draw[step=0.5cm,black,very thin] (-2,-2) grid (2,2);
		\end{scope}
		
		\matrix[matrix, matrix of nodes, nodes={anchor=center,inner sep=0pt,text width=.5cm,align=center,minimum height=.5cm}, nodes in empty cells]{
			& 0 & 1 & 2 & 3 & 4 & 5 & 6 & 7 \\
			0 &   &   &   &   &   &   &   &   \\
			1 &   &   &   &   &   &   &   &   \\
			2 &   &   &   &   &   &   &   &   \\
			3 &   &   &   &   &   &   &   &   \\
			4 & A & 1 & 2 & 3 & 4 & 5 & 6 & B \\
			5 &   &   &   &   &   &   &   &   \\
			6 &   &   &   &   &   &   &   &   \\
			7 &   &   &   &   &   &   &   &   \\
		};
	\end{scope}
\end{tikzpicture}
	\caption{Usporedba pretraživanja u širinu i optimalnog rješenja.} 
	\label{inefficient_bfs}
\end{figure}

