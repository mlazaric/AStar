Pretraživanje u širinu \engl{breadth-first search} je jednostavan algoritam za pronalazak najkraćeg puta u grafu s uniformnim težinama bridova.
Algoritam je potpun i optimalan, a njegova vremenska i memorijska složenost je \( O(b^d) \).

Za pohranu vrhova koje treba obraditi koristi red \engl{queue}, dok za pohranu već obrađenih vrhova koristi skup \engl{set}.
U svakom koraku širi se u svim mogućim smjerovima, što dovodi do velike memorijske i vremenske složenosti.
Implementiran je funkcijom \texttt{bfs} u razredu \texttt{SearchAlgorithms}.

Na slici \ref{inefficient_bfs} je prikazano izvođenje algoritma.
Crvenom bojom su označena polja koje je algoritam obradio, ali nisu dovela do najkraćeg puta, dok je zelenom označen najkraći put.
Brojevi u poljima označavaju udaljenost od početnog polja.
Pretraga u širinu je obradila \( 44 \) polja, dok bi optimalno rješenje obradilo samo \( 8 \). 

\begin{figure}[h]
	\centering
	\input{figures/inefficient_bfs.tex}
	\caption{Usporedba pretraživanja u širinu i optimalnog rješenja.} 
	\label{inefficient_bfs}
\end{figure}

