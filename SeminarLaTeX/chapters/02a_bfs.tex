Pretraga u širinu \engl{breadth-first search} je jednostavan algoritam za pronalazak najkraćeg puta u grafu s uniformnim težinama bridova. Za pohranu vrhova koje treba obraditi koristi red \engl{queue}, dok za pohranu već obrađenih vrhova koristi skup \engl{set}. U svakom koraku, širi se u svim mogućim smjerovima, što dovodi do neefikasnosti samog algoritma.
\todo[inline]{Spomenuti da je BFS dobar generalni algoritam, ali da je zato vremenski i memorijski zahtjevan.}

Na slici \ref{inefficient_bfs} je vidljiva ta neefikasnost.
Crvenom bojom su označena polja koje je algoritam obradio, ali nisu dovela do najkraćeg puta, dok je zelenom označen najkraći put.
Brojevi u poljima označavaju udaljenost od početnog polja.
Pretraga u širinu je obradila \( 44 \) polja, dok bi optimalni algoritam obradio samo \( 8 \). 

\todo[inline]{Spomenuti vremensku i memorijsku složenost.}

\begin{figure}[h]
	\centering
	\input{figures/inefficient_bfs.tex}
	\caption{Usporedba pretrage u širinu i optimalnog algoritma.} 
	\label{inefficient_bfs}
\end{figure}

\begin{listing}[H]
	\inputminted{python}{code_listings/pretraga_u_sirinu.py}
	\caption{Pretraga u širinu implementirana u Pythonu.}
	\label{pretraga_u_sirinu}
\end{listing}

\begin{listing}[H]
	\inputminted{python}{code_listings/procitaj_put.py}
	\caption{Pomoćna funkcija za čitanje najkraćeg puta.}
	\label{procitaj_put}
\end{listing}

