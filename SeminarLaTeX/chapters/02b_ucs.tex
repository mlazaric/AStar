Pretraživanje s jednolikom cijenom \engl{uniform-cost search} je jednostavan algoritam za pronalazak najkraćeg puta u grafu s pozitivnim težinama bridova.
Algoritam je potpun i optimalan, a njegova vremenska i memorijska složenost je \( O \left (b^{1 + \lfloor {C^*}/{\epsilon} \rfloor} \right ) \) gdje \( C^* \) predstavlja cijenu optimalnog rješenja, a \( \epsilon \) minimalnu težinu svih bridova. \cite{russelNorvig2003:aima}

Za pohranu vrhova koristi prioritetni red \engl{priority queue}, te hash tablicu radi efikasne provjere je li pronašao bolji put do nekog vrha koji se već nalazi u prioritetnom redu.
Implementiran je metodom \texttt{ucs} u razredu \texttt{SearchAlgorithms}.

Prioritetni red koristi evaluacijsku funkciju \engl{evaluation function} za određivanje prioriteta, pri tome se manji broj smatra većim prioritetom. 
Označava se s \( f(n) \).
Konkretno za pretraživanje s jednolikom cijenom evaluacijska funkcija je jednaka cijenom puta od početnog vrha do tog vrha. 
Oznaka za cijenu puta od početka do tog vrha je \( g(n) \), pa za pretraživanje s jednolikom cijenom vrijedi \( f(n) = g(n) \).

\todo[inline]{Pseudokod}

Za grafove s uniformnim težinama bridova, pretraživanje s jednolikom cijenom obradi strogo više vrhova od pretraživanja u širinu jer provjerava je li vrh rješenje pri obradi vrha.
Posljedica toga je da algoritam obrađuje sve vrhove koji imaju istu cijenu, prije nego što obradi njihove sljedbenike.

\begin{figure}[h]
	\centering
	\begin{tikzpicture}
		\begin{scope}
			\begin{scope}[xshift=0.25cm, yshift=-0.25cm]
	\fill[fill = green!10] (-2, -0.5) rectangle (2, 0);
	
	\path[fill = red!10] (-2, -0.5) -- (1, -0.5) -- (1, -1) -- (0.5, -1) -- (0.5, -1.5) -- (0, -1.5) -- (0, -2) -- (-2, -2) -- cycle;
	
	\path[fill = red!10] (-2, 0) -- ++(0, 2) -- ++(1.5, 0) -- ++(0, -0.5) -- ++(0.5, 0) -- ++(0, -0.5) -- ++(0.5, 0) -- ++(0, -0.5) -- ++(0.5, 0) -- ++(0, -0.5) -- cycle;

	\draw[step=0.5cm,black,very thin] (-2,-2) grid (2,2);
\end{scope}

\matrix[matrix, matrix of nodes, nodes={anchor=center,inner sep=0pt,text width=.5cm,align=center,minimum height=.5cm}, nodes in empty cells]{
	  & 0 & 1 & 2 & 3 & 4 & 5 & 6 & 7 \\
	0 & 4 & 5 & 6 &   &   &   &   &   \\
	1 & 3 & 4 & 5 & 6 &   &   &   &   \\
	2 & 2 & 3 & 4 & 5 & 6 &   &   &   \\
	3 & 1 & 2 & 3 & 4 & 5 & 6 &   &   \\
	4 & A & 1 & 2 & 3 & 4 & 5 & 6 & B \\
	5 & 1 & 2 & 3 & 4 & 5 & 6 &   &   \\
	6 & 2 & 3 & 4 & 5 & 6 &   &   &   \\
	7 & 3 & 4 & 5 & 6 &   &   &   &   \\
};
		\end{scope}
		
		\begin{scope}[xshift = 7.5cm]
			
\begin{scope}[xshift=0.25cm, yshift=-0.25cm]
	\fill[fill = green!10] (-2, -0.5) rectangle (2, 0);
	
	\path[fill = red!10] (-2, -0.5) -- (1.5, -0.5) -- (1.5, -1) -- (1, -1) -- (1, -1.5) -- (0.5, -1.5) -- (0.5, -2) -- (-2, -2) -- cycle;
	
	\path[fill = red!10] (-2, 0) -- ++(0, 2) -- ++(2, 0) -- ++(0, -0.5) -- ++(0.5, 0) -- ++(0, -0.5) -- ++(0.5, 0) -- ++(0, -0.5) -- ++(0.5, 0) -- ++(0, -0.5) -- cycle;
	
	\draw[step=0.5cm,black,very thin] (-2,-2) grid (2,2);
\end{scope}

\matrix[matrix, matrix of nodes, nodes={anchor=center,inner sep=0pt,text width=.5cm,align=center,minimum height=.5cm}, nodes in empty cells]{
	  & 0 & 1 & 2 & 3 & 4 & 5 & 6 & 7 \\
	0 & 4 & 5 & 6 & 7 &   &   &   &   \\
	1 & 3 & 4 & 5 & 6 & 7 &   &   &   \\
	2 & 2 & 3 & 4 & 5 & 6 & 7 &   &   \\
	3 & 1 & 2 & 3 & 4 & 5 & 6 & 7 &   \\
	4 & A & 1 & 2 & 3 & 4 & 5 & 6 & B \\
	5 & 1 & 2 & 3 & 4 & 5 & 6 & 7 &   \\
	6 & 2 & 3 & 4 & 5 & 6 & 7 &   &   \\
	7 & 3 & 4 & 5 & 6 & 7 &   &   &   \\
};
		\end{scope}
	\end{tikzpicture}
	\caption{Usporedba pretraživanja u širinu i pretraživanja s jednolikom cijenom.} 
	\label{inefficient_ucs}
\end{figure}