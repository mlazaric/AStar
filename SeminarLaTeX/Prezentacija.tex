\documentclass{beamer}

\usepackage[utf8]{inputenc}
\usepackage[T1]{fontenc}
\usepackage{tikz}
\usepackage{fontspec}
\usepackage{wrapfig}
\usepackage{float}
\usepackage{amsmath}

\usetheme{metropolis}
\usetikzlibrary{matrix}
\setmainfont[Ligatures=TeX]{Fira Sans}

\title{Pronalazak najkraćeg puta algoritmom A*}

\date{\today}

\author{Marko Lazarić\\
Voditelj: Doc. dr. sc. Marko Čupić}

\newcommand{\engl}[1]{ (engl. \emph{#1})}
\newcommand{\twocolumns}[2]{
	\begin{columns}
		\begin{column}{0.4\textwidth}
			#1
		\end{column}
		\begin{column}{0.6\textwidth}  %%<--- here
			#2
		\end{column}
	\end{columns}
}
\newcommand{\state}[2]{\textsc{Stanje}(\( #1 \), \( #2 \))}

\newcommand{\actionName}[1]{#1}
\newcommand{\upAction}{\actionName{Gore}}
\newcommand{\downAction}{\actionName{Dolje}}
\newcommand{\leftAction}{\actionName{Lijevo}}
\newcommand{\rightAction}{\actionName{Desno}}

\begin{document}
  \maketitle
  
  \section{Definicija problema}
  \begin{frame}{Početno stanje}	
  	\twocolumns{
	  \begin{figure}[H]
	  		\begin{tikzpicture}
	  		\input{figures/initial_state.tex}
	  		\end{tikzpicture}
	  \end{figure}
    }{
      \begin{itemize}
    	\item \emph{Početno stanje} \engl{initial state} je stanje u kojem počinje pretraživanje
      	\item U slučaju cjelobrojne rešetke, to je stanje koje sadrži "A"
      \end{itemize}
    }
  \end{frame}

  \begin{frame}{Akcije}
    \twocolumns{
    	\begin{figure}[H]
    		\begin{tikzpicture}
    		\input{figures/actions.tex}
    		\end{tikzpicture}
    	\end{figure}
    }{
	    \begin{itemize}
	    	\item \emph{Akcije} \engl{actions} je popis mogućih akcija iz trenutnog stanja
	    	\item U slučaju cjelobrojne rešetke, moguće akcije su pomakni se gore, dolje, lijevo ili desno
	    \end{itemize}
    }
  \end{frame}

  \begin{frame}{Model prijelaza}
    \begin{itemize}
    	\item \emph{Model prijelaza} \engl{transition model} opisuje moguće akcije i prijelaze između stanja
    	\item U slučaju cjelobrojne rešetke, model prijelaza je:
    	\\[1em]
    	\begin{tabular}{lcl}
    		\textsc{Rezultat}(\state{x}{y}, \upAction) &=& \state{x}{y - 1} \\
    		\textsc{Rezultat}(\state{x}{y}, \downAction) &=& \state{x}{y + 1} \\
    		\textsc{Rezultat}(\state{x}{y}, \leftAction) &=& \state{x - 1}{y} \\
    		\textsc{Rezultat}(\state{x}{y}, \rightAction) &=& \state{x + 1}{y}
    	\end{tabular}
    	%\begin{itemize}
    	%	\item \textsc{Rezultat}(Stanje(x, y), Gore) = Stanje(x, y - 1)
    	%	\item \textsc{Rezultat}(Stanje(x, y), Dolje) = Stanje(x, y + 1)
    	%	\item \textsc{Rezultat}(Stanje(x, y), Lijevo) = Stanje(x - 1, y)
    	%	\item \textsc{Rezultat}(Stanje(x, y), Desno) = Stanje(x + 1, y)
    	%\end{itemize}
    \end{itemize}
  \end{frame}

  \begin{frame}{Provjera rješenja}
  	\twocolumns{
    	\begin{figure}[H]
  			\begin{tikzpicture}
	  			\input{figures/goal_test.tex}
  			\end{tikzpicture}
  		\end{figure}
  	}{
	  	\begin{itemize}
	  		\item \emph{Provjera rješenja} \engl{goal test} provjerava je li određeno stanje rješenje
	  		\item U slučaju cjelobrojne rešetke, stanje je rješenje ako sadrži "B"
	  	\end{itemize}
    }
  \end{frame}

  \begin{frame}{Trošak prijelaza između stanja}
	\begin{itemize}
      \item \emph{Trošak prijelaza} \engl{step cost} je funkcija koja određuje trošak prijelaza iz stanja \( s \) u stanje \( s' \) akcijom \( a \)
      \item Označava se \( c(s, a, s') \)
	  \item U slučaju cjelobrojne rešetke, trošak prijelaza je \( 1 \) za svako susjedno polje (osim zidova)
    \end{itemize}
  \end{frame}

  \begin{frame}{Heuristička funkcija - definicija}
  	\begin{itemize}
  	  \item \emph{Heuristička funkcija} \engl{heuristic} predstavlja najmanju cijenu puta od stanja \( n \) do stanja koje predstavlja rješenje
  	  \item Označava se \( h(n) \)
  	  \item Za svako stanje \( n \) koje predstavlja rješenje, mora vrijediti \( h(n) = 0 \)
  	\end{itemize}
  \end{frame}

  \begin{frame}{Heuristička funkcija - primjer}
    \twocolumns{
    	\begin{figure}[H]
    		\begin{tikzpicture}
    		\input{figures/heuristic.tex}
    		\end{tikzpicture}
    	\end{figure}
    }{
    	\begin{itemize}
    		\item Jednostavna heuristika za cjelobrojnu rešetku je Manhattan udaljenost između stanja \\[0.5cm]
    	\end{itemize}
    	\[ h(\text{\state{x}{y}}) = |x - x_B| + |y - y_B| \]
    	\[ h(\text{\state{0}{0}}) = 7 \]
    	\[ h(\text{\state{4}{0}}) = 11 \]
    }
  \end{frame}
\end{document}